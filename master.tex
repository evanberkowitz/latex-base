\documentclass[aps,superscriptaddress,tightenlines,nofootinbib,floatfix,longbibliography]{revtex4-1}
\usepackage[left=18mm,right=19mm,top=23mm,bottom=16mm]{geometry}
\usepackage{amsmath,amssymb}
\usepackage{bm}
\usepackage{comment}
\usepackage{graphicx}
\usepackage[dvipsnames]{xcolor}
\usepackage{slashed}
\usepackage[
    colorlinks=true,
    allcolors=blue
]{hyperref}



%%%
%%%     Draft margin.
%%%

\usepackage[
    angle=90,
    color=black,
    opacity=1,
    scale=2,
    ]{background}
\SetBgPosition{current page.west}
\SetBgVshift{-4.5mm}
\backgroundsetup{contents={\input{git_information}}}

\usepackage{xspace}
\usepackage{bbm}

%%%%
%%%%    Referring to Parts of the Document
%%%%

\newcommand{\tabref}[1]{Tab.~\ref{tab:#1}\xspace}
\newcommand{\Tabref}[1]{Table~\ref{tab:#1}\xspace}
\newcommand{\figref}[1]{Fig.~\ref{fig:#1}\xspace}
\newcommand{\Figref}[1]{Figure~\ref{fig:#1}\xspace}
\renewcommand{\eqref}[1]{(\ref{#1})\xspace}
\newcommand{\Eqref}[1]{Equation~\ref{eq:#1}\xspace}

%%%%
%%%%    Mathematical Symbols
%%%%

\newcommand{\goesto}{\ensuremath{\rightarrow}}
\newcommand{\one}{\ensuremath{\mathbbm{1}}}

%%%%
%%%%    Physical Quantities and Constants
%%%%


%%%%
%%%%    Software
%%%%

\newcommand{\git}{\texttt{git}\xspace}

% Put an xspace after \LaTeX:
\let\builtinLaTeX\LaTeX
\def\LaTeX{\builtinLaTeX\xspace}
 % input rather than include so we don't create macros.aux

%%%%
%%%%    Document preparation
%%%%


\begin{document}

\title{Example \LaTeX + \texttt{git} }

\author{Evan Berkowitz}

\date{\today}

\begin{abstract}
Here's a bare-bones example where I set up a \git repo with all the \LaTeX-related things that I like.
\vfill % And below is where you see the state of the repo that produced this pdf.
\input{git_information}
\end{abstract}

\maketitle

\section{Introduction}\label{sec:intro}

\LaTeX is a system for producing beautiful documents.
It takes as input a set of plain text files, and produces a variety of output types; the most useful to me is PDF.

\git is a decentralized version control system.
It's fantastic.
There is a version control package, \texttt{vc}\cite{vc} which provides some of the same functionality provided here; you may be better off with that, depending on your use case.

This repo\cite{latex-base} is an example that is supposed to make it easy to make \git-controlled \LaTeX documents.
In particular, I have written \git hooks that ensure individual commits successfully produce a PDF and ensure that after a pull the PDF is automatically recompiled with the latest edits.
That helps keep everybody\ldots\ on the same page.
\section{\texttt{git} Hooks}\label{sec:hooks}

\git hooks are scripts that you inject into the \git workflow.
They can be installed into the repo with
\begin{verbatim}
    make git-hooks
\end{verbatim}
which symbolically links the hooks in the \texttt{hooks} directory into the \texttt{.git/hooks} directory.

The \texttt{pre-commit} hook prevents the user from \texttt{git commit}ting a commit that doesn't compile, which it tests for by stashing all files that don't belong to the commit and trying to \texttt{make master.pdf}.
The output of that \make is created in the \texttt{.git-pre-commit-hook.log} in the root directory of this repository, so that the user can see the \LaTeX in the event the PDF doesn't compile.

The \texttt{post-merge} hook tries to compile the latest PDF, also using \texttt{make master.pdf} after the repository has been updated.
It is much less sophisticated than the pre-commit hook, because (under the assumption the pre-commit hook is working) the committed state of the repo should always compile.

\section{Makefile}

The provided makefile uses \texttt{pdflatex} and \texttt{bibtex} to compile \texttt{master.pdf}, the base of which is set in the MASTER variable.
You can change the executables in the \texttt{TEX} and \texttt{BIB} variables.

The default target is \texttt{\$(MASTER).pdf}, drawn from \texttt{\$(MASTER).tex}.
You can add other root-level documents and simply do \texttt{make other-document.pdf}.
The PDF targets depend on the files in the \texttt{section} directory, \texttt{macros.tex}, all \texttt{.bib} files in the repository, and the corresponding root-level \TeX\ document.

The \texttt{git-hooks} target symbolically links the git hooks into the repository's \texttt{.git} directory.

The \texttt{tidy} and \texttt{clean} targets remove a lot of cruft automatically generated during the course of compiling the \LaTeX to PDF.

The \texttt{watch} target uses \texttt{watchman-make}\cite{watchman} to continuously update the PDF as you make changes to the source files.
This can interfere with the pre-commit git hook.
It can also get stuck in an infinite loop.
But if you tend to write decent \LaTeX the first time around, it'll be pretty reliable.
It is sensitive to the \texttt{DRAFT} variable.
If you want to watch and recompile a different PDF, you can set \texttt{TARGET=pdf-file-prefix}, leaving out the \texttt{.pdf}.
The \texttt{TARGET} variable is also used by the \texttt{tidy} and \texttt{clean} targets.
It is automatically set to \texttt{\$(MASTER)} if not otherwise specified.

The \texttt{DRAFT} variable also controls whether the \git repository information is produced on the left margin, as explained in the introduction.


\bibliography{master}

\end{document}