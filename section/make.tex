\section{Makefile}

The provided makefile uses \texttt{pdflatex} and \texttt{bibtex} to compile \texttt{master.pdf}, the base of which is set in the MASTER variable.

The \texttt{\$(MASTER).pdf} target depends on the files in the \texttt{section} directory, \texttt{macros.tex}, and \texttt{\$(MASTER).tex}.

The \texttt{\$(GIT\_STATUS)} phony target uses \texttt{git\_information.sh} to produce the \texttt{git\_information.tex} file explained above.

The \texttt{git-hooks} target symbolically links the git hooks into the repository's \texttt{.git} directory.

The \texttt{clean\_temporary\_files} and \texttt{clean} targets remove a lot of cruft automatically generated during the course of compiling the \LaTeX to PDF.

The \texttt{watch} target uses \texttt{watchman-make}\cite{watchman} to continuously update the PDF as you make changes to the source files.
This can interfere with the pre-commit git hook.
It can also get stuck in an infinite loop.
But if you tend to write decent \LaTeX the first time around, it'll be pretty reliable.

