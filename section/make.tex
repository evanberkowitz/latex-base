\section{Makefile}

The provided makefile uses \texttt{pdflatex} and \texttt{bibtex} to compile \texttt{master.pdf}, the base of which is set in the MASTER variable.
You can change the executables in the \texttt{TEX} and \texttt{BIB} variables.

The default target is \texttt{\$(MASTER).pdf}, drawn from \texttt{\$(MASTER).tex}.
You can add other root-level documents and simply do \texttt{make other-document.pdf}.
The PDF targets depend on the files in the \texttt{section} directory, \texttt{macros.tex}, all \texttt{.bib} files in the repository, and the corresponding root-level \TeX\ document.

The \texttt{git-hooks} target symbolically links the git hooks into the repository's \texttt{.git} directory.

The \texttt{tidy} and \texttt{clean} targets remove a lot of cruft automatically generated during the course of compiling the \LaTeX to PDF.

The \texttt{watch} target uses \texttt{watchman-make}\cite{watchman} to continuously update the PDF as you make changes to the source files.
This can interfere with the pre-commit git hook.
It can also get stuck in an infinite loop.
But if you tend to write decent \LaTeX the first time around, it'll be pretty reliable.
It is sensitive to the \texttt{DRAFT} variable.
If you want to watch and recompile a different PDF, you can set \texttt{TARGET=pdf-file-prefix}, leaving out the \texttt{.pdf}.
The \texttt{TARGET} variable is also used by the \texttt{tidy} and \texttt{clean} targets.
It is automatically set to \texttt{\$(MASTER)} if not otherwise specified.

The \texttt{DRAFT} variable also controls whether the \git repository information is produced on the left margin, as explained in the introduction.

The \texttt{diff} target uses \texttt{git latexdiff} to produce a diff\cite{git-latexdiff}.