\section{Introduction}

\LaTeX is a system for producing beautiful documents.
It takes as input a set of plain text files, and produces a variety of output types; the most useful to me is PDF.

\git is a decentralized version control system.
It's fantastic.

This repo is an example that is supposed to make it easy to make \git-controlled \LaTeX documents.
In particular, I have written \git hooks that ensure individual commits successfully produce a PDF and ensure that after a pull the PDF is automatically recompiled with the latest edits.
That helps keep everybody\ldots\ on the same page.

I also have a small shell script that reports the status of the repo when the PDF is being compiled that can be immediately incorporated into the \LaTeX document.
The script \texttt{git\_information.sh} produces \texttt{git\_information.tex} which looks, in the current example, like
\begin{verbatim}
% Automatically generated by git_information.sh
\newcommand{\gitRevision}{9c35867408bd1e7915f9f7501ac341fa497834f1}
\newcommand{\gitRevisionDate}{2018-06-08 14:07:44 +0200}
\newcommand{\gitDirtyFiles}{1}
\newcommand{\gitInfo}{\gitDirtyFiles\ files different from commit \gitRevision\ from \gitRevisionDate}
\end{verbatim}
The one dirty file is the one I'm working on---\texttt{section/introduction.tex}.
In \texttt{master.tex}, I have \texttt{\textbackslash include\{git\_information\}} and invoke \texttt{\textbackslash gitInfo} at the end of the abstract, which produces the git status at the bottom of the cover page of this document.
