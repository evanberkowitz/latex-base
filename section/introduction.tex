\section{Introduction}

\LaTeX is a system for producing beautiful documents.
It takes as input a set of plain text files, and produces a variety of output types; the most useful to me is PDF.

\git is a decentralized version control system.
It's fantastic.
There is a version control package, \texttt{vc}\cite{vc} which provides some of the same functionality provided here; you may be better off with that, depending on your use case.

This repo\cite{latex-base} is an example that is supposed to make it easy to make \git-controlled \LaTeX documents.
In particular, I have written \git hooks that ensure individual commits successfully produce a PDF and ensure that after a pull the PDF is automatically recompiled with the latest edits.
That helps keep everybody\ldots\ on the same page.

I also have a small shell script that reports the status of the repo when the PDF is being compiled.
The script \texttt{repo/git.sh} produces some \LaTeX which, in the current example, like
\begin{verbatim}
\newcommand{\repositoryInformationSetup}{
    \usepackage[dvipsnames]{xcolor}
    \usepackage[ angle=90, color=black, opacity=1, scale=2, ]{background}
    \SetBgPosition{current page.west}
    \SetBgVshift{-4.5mm}
    \backgroundsetup{contents={{\color{Red}1 file different from 
        commit da2aebc from 2018-11-05 15:29:15 +0100}}}
}\end{verbatim}
The one dirty file is the one I'm working on---\texttt{section/introduction.tex}.
To get the \git status on the left margin of every page I simply invoke \texttt{make DRAFT=1}.
One may invoke \make alone or use, for example, \texttt{pdflatex master.tex}, to produce a PDF without the \git information in the margin.

By adding additional scripts in the \texttt{repo} directory and changing the \texttt{REPO} makefile variable from \git to a version control system of your choice, you can extend the ability of this skeleton to provide information about the state of the repository.
The script must simply define \texttt{{\textbackslash}repositoryInformationSetup} which is invoked once in the preamble to \texttt{master.tex}.
