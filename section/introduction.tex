\section{Introduction}

\LaTeX is a system for producing beautiful documents.
It takes as input a set of plain text files, and produces a variety of output types; the most useful to me is PDF.

\git is a decentralized version control system.
It's fantastic.

This repo is an example that is supposed to make it easy to make \git-controlled \LaTeX documents.
In particular, I have written \git hooks that ensure individual commits successfully produce a PDF and ensure that after a pull the PDF is automatically recompiled with the latest edits.
That helps keep everybody\ldots\ on the same page.

I also have a small shell script that reports the status of the repo when the PDF is being compiled that can be immediately incorporated into the \LaTeX document.
The script \texttt{git\_information.sh} produces \texttt{git\_information.tex} which looks, in the current example, like
\begin{verbatim}
% Automatically generated by git_information.sh
1 files different from commit d50192b from 2018-08-24 10:26:28 +0200
\end{verbatim}
The one dirty file is the one I'm working on---\texttt{section/introduction.tex}.
To get the \git status I simply do \texttt{\textbackslash{}input\{git\_information\}}.
This is included in the left margin of the \texttt{master.pdf} if \make is invoked with \texttt{DRAFT=1}.
If the \texttt{DRAFT} variable remains undefined \texttt{git\_information.tex} is set to the empty file.
